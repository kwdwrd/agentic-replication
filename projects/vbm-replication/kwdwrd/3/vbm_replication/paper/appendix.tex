\documentclass[12pt]{article}

\usepackage[utf8]{inputenc}
\usepackage[T1]{fontenc}
\usepackage{amsmath,amssymb}
\usepackage{graphicx}
\usepackage{booktabs}
\usepackage{hyperref}
\usepackage{setspace}
\usepackage[margin=1in]{geometry}
\usepackage{float}
\usepackage{longtable}
\usepackage{threeparttable}

\title{Online Appendix: \\
Universal Vote-by-Mail and Partisan Vote Share: \\
Replication and Extension Through 2020}

\date{\today}

\begin{document}

\maketitle
\doublespacing

\appendix

\section{Data Sources}

\subsection{Original Data}

The original Thompson et al. (2020) replication materials are available at:
\begin{verbatim}
https://github.com/stanford-dpl/vbm
\end{verbatim}

The analysis dataset (\texttt{analysis.dta}) contains 1,454 observations for 126 counties across California (58), Utah (29), and Washington (39) from 1996--2018.

\subsection{Extension Data}

Extension data were collected from the following sources:

\begin{itemize}
    \item \textbf{2020 Presidential Results}: MIT Election Data + Science Lab county-level returns
    \item \textbf{California VCA Adoption}: California Secretary of State, League of Women Voters of California
    \item \textbf{Utah/Washington Treatment Status}: State election office records
\end{itemize}

\section{California VCA Adoption Details}

Table \ref{tab:vca_full} provides the complete list of California VCA adopters by year.

\begin{longtable}{lll}
\caption{California Voter's Choice Act Adoption by County} \label{tab:vca_full} \\
\toprule
County & First VCA Year & Population (2020) \\
\midrule
\endfirsthead
\multicolumn{3}{c}{\tablename\ \thetable\ -- continued from previous page} \\
\toprule
County & First VCA Year & Population (2020) \\
\midrule
\endhead
\midrule
\multicolumn{3}{r}{Continued on next page} \\
\endfoot
\bottomrule
\endlastfoot
% 2018 Adopters
Madera & 2018 & 157,327 \\
Napa & 2018 & 137,744 \\
Nevada & 2018 & 99,755 \\
Sacramento & 2018 & 1,552,058 \\
San Mateo & 2018 & 764,442 \\
\midrule
% 2020 Adopters
Amador & 2020 & 40,474 \\
Butte & 2020 & 219,186 \\
Calaveras & 2020 & 45,905 \\
El Dorado & 2020 & 192,843 \\
Fresno & 2020 & 999,101 \\
Los Angeles & 2020 & 10,014,009 \\
Mariposa & 2020 & 17,131 \\
Orange & 2020 & 3,186,989 \\
Santa Clara & 2020 & 1,927,852 \\
Tuolumne & 2020 & 54,478 \\
\midrule
% 2022 Adopters
Alameda & 2022 & 1,671,329 \\
Contra Costa & 2022 & 1,153,526 \\
Marin & 2022 & 258,826 \\
Merced & 2022 & 281,202 \\
Riverside & 2022 & 2,470,546 \\
San Benito & 2022 & 64,209 \\
San Bernardino & 2022 & 2,181,654 \\
San Diego & 2022 & 3,298,634 \\
San Francisco & 2022 & 873,965 \\
Sonoma & 2022 & 494,336 \\
Ventura & 2022 & 843,843 \\
Yolo & 2022 & 220,500 \\
\midrule
% 2024 Adopters
Placer & 2024 & 404,739 \\
San Luis Obispo & 2024 & 283,111 \\
\end{longtable}

\section{Replication Details}

\subsection{Software}

Replication and extension analyses were conducted using:
\begin{itemize}
    \item Python 3.12
    \item pandas 2.1.0
    \item pyfixest 0.14.0
    \item statsmodels 0.14.0
\end{itemize}

Original analyses used Stata 16 with the \texttt{reghdfe} package.

\subsection{Replication Comparison}

Table \ref{tab:replication_detail} provides detailed comparison of original and replicated estimates.

\begin{table}[H]
\centering
\caption{Detailed Replication Comparison}
\label{tab:replication_detail}
\begin{threeparttable}
\begin{tabular}{lccccc}
\toprule
Outcome & Original & Replicated & Difference & $N$ & Match \\
\midrule
\multicolumn{6}{l}{\textit{Table 2: Democratic Vote Share}} \\
Presidential (basic) & 0.006 & 0.006 & 0.000 & 378 & \checkmark \\
Gubernatorial (basic) & $-$0.006 & $-$0.006 & 0.000 & 438 & \checkmark \\
\midrule
\multicolumn{6}{l}{\textit{Table 3: Turnout}} \\
Turnout share (basic) & 0.024 & 0.024 & 0.000 & 986 & \checkmark \\
\bottomrule
\end{tabular}
\begin{tablenotes}
\small
\item Notes: ``Basic'' refers to specification with county and state-by-year fixed effects without county-specific trends.
\end{tablenotes}
\end{threeparttable}
\end{table}

\section{Extension Results: Full Tables}

\subsection{Event Study}

Table \ref{tab:event_study} presents Democratic vote share by event time (years relative to VCA adoption) for California counties that adopted VCA.

\begin{table}[H]
\centering
\caption{Event Study: Democratic Vote Share by Event Time}
\label{tab:event_study}
\begin{tabular}{lccc}
\toprule
Event Time & Mean Dem. Share & Std. Dev. & $N$ Counties \\
\midrule
$t = -22$ & 0.558 & --- & 12 \\
$t = -20$ & 0.460 & --- & 12 \\
$t = -18$ & 0.539 & --- & 17 \\
$t = -16$ & 0.462 & --- & 12 \\
$t = -14$ & 0.598 & --- & 17 \\
$t = -12$ & 0.512 & --- & 12 \\
$t = -10$ & 0.613 & --- & 17 \\
$t = -8$ & 0.502 & --- & 12 \\
$t = -6$ & 0.630 & --- & 17 \\
$t = -4$ & 0.513 & --- & 12 \\
$t = -2$ & 0.642 & --- & 17 \\
$t = 0$ & 0.508 & --- & 10 \\
$t = +2$ & 0.629 & --- & 5 \\
\bottomrule
\end{tabular}
\end{table}

Note: Event time reflects presidential election years only. Even-numbered event times correspond to presidential years; odd-numbered event times (not shown) correspond to midterm years.

\subsection{State-Specific Results}

Table \ref{tab:state_specific} presents separate analyses for each state.

\begin{table}[H]
\centering
\caption{State-Specific Extension Results}
\label{tab:state_specific}
\begin{threeparttable}
\begin{tabular}{lcccc}
\toprule
State & Coefficient & Std. Error & $N$ & Treatment Variation \\
\midrule
California & 0.002 & 0.010 & 348 & VCA adoption 2018--2024 \\
Utah & --- & --- & 174 & No variation (all treated 2020+) \\
Washington & --- & --- & 234 & No variation (all treated 2011+) \\
\midrule
All states & 0.017 & 0.008 & 756 & --- \\
\bottomrule
\end{tabular}
\begin{tablenotes}
\small
\item Notes: Utah and Washington estimates not reported due to lack of treatment variation in the extension period. All counties in Utah and Washington were treated by 2020.
\end{tablenotes}
\end{threeparttable}
\end{table}

\section{Data Limitations}

\subsection{Missing Extension Data}

The following data were unavailable for the extension analysis:

\begin{enumerate}
    \item \textbf{2022 Gubernatorial Results}: California and Utah held gubernatorial elections in 2022, but county-level results were not obtained due to data access restrictions.

    \item \textbf{2024 Presidential Results}: The 2024 presidential election occurred after the extension data collection period.

    \item \textbf{CVAP Data}: Citizen Voting Age Population estimates from the Census Bureau were not obtained for the extension period, precluding turnout analysis.

    \item \textbf{Party Registration}: Washington State does not require party registration, limiting analysis of partisan turnout composition.
\end{enumerate}

\subsection{Implications for Analysis}

Due to these limitations, our extension analysis focuses on:
\begin{itemize}
    \item Democratic presidential vote share as the primary outcome
    \item The 2020 presidential election as the primary extension year
    \item California as the primary source of treatment variation
\end{itemize}

Future research should extend the analysis through 2024 once complete data become available.

\end{document}
