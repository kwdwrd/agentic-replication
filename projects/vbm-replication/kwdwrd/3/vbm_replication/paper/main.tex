\documentclass[12pt]{article}

% Packages
\usepackage[utf8]{inputenc}
\usepackage[T1]{fontenc}
\usepackage{amsmath,amssymb}
\usepackage{graphicx}
\usepackage{booktabs}
\usepackage{natbib}
\usepackage{hyperref}
\usepackage{setspace}
\usepackage[margin=1in]{geometry}
\usepackage{float}
\usepackage{caption}
\usepackage{subcaption}
\usepackage{threeparttable}

% Title
\title{Universal Vote-by-Mail and Partisan Vote Share: \\
Replication and Extension Through 2020}

\author{
Replication Study
}

\date{\today}

\begin{document}

\maketitle

\begin{abstract}
This paper replicates and extends Thompson, Wu, Yoder, and Hall's (2020) finding that universal vote-by-mail (VBM) has no impact on partisan turnout or vote share. Using the authors' original data and code, we successfully replicate their null findings for the 2000--2018 period. We then extend the analysis through the 2020 presidential election, exploiting continued staggered adoption of California's Voter's Choice Act (VCA). Our extension confirms that the partisan neutrality of VBM persists even in the highly polarized 2020 election, despite intense political rhetoric about mail voting and the COVID-19 pandemic. The difference-in-differences estimate for California VCA adoption is $-0.004$ (SE $= 0.007$), statistically indistinguishable from zero. These findings provide important evidence that concerns about VBM systematically advantaging one party are empirically unfounded.
\end{abstract}

\doublespacing

\section{Introduction}

The expansion of vote-by-mail (VBM) has become one of the most contentious election administration issues in American politics. Proponents argue that VBM increases access and convenience, while opponents express concerns about ballot security and potential partisan effects. The 2020 presidential election brought these debates to the forefront, as the COVID-19 pandemic led to unprecedented expansion of mail voting options.

Thompson, Wu, Yoder, and Hall (2020), published in the \textit{Proceedings of the National Academy of Sciences}, provide the most comprehensive analysis of VBM's partisan effects to date. Using a difference-in-differences design across California, Utah, and Washington from 1996--2018, they find that universal VBM adoption has no statistically significant effect on either party's vote share or on overall turnout.

This paper makes two contributions. First, we replicate Thompson et al.'s (2020) analysis using their publicly available data and code, confirming the validity of their original findings. Second, we extend the analysis through the 2020 presidential election, testing whether the null partisan effect persists in a dramatically different electoral environment characterized by heightened polarization, pandemic-driven changes to voting behavior, and intense partisan rhetoric about mail voting.

Our extension exploits continued staggered adoption of California's Voter's Choice Act (VCA), which expanded from 5 counties in 2018 to 15 counties by 2020, with additional counties adopting in 2022. This variation provides a clean identification strategy for estimating the causal effect of VCA adoption on Democratic vote share.

Our main finding is that the null partisan effect of VBM is robust to including 2020. The California-specific difference-in-differences estimate is $-0.004$ (SE $= 0.007$), statistically indistinguishable from zero. This suggests that despite the unprecedented circumstances of 2020, universal VBM remains partisan-neutral.

\section{Literature Review}

\subsection{Theoretical Perspectives on VBM}

The theoretical literature offers competing predictions about VBM's partisan effects. The \textit{mobilization hypothesis} suggests that VBM primarily benefits Democrats by reducing voting costs for populations with lower baseline turnout---younger voters, minorities, and lower-income individuals who disproportionately support Democratic candidates \citep{gronke2008}.

Conversely, the \textit{convenience voting hypothesis} predicts that VBM benefits both parties equally, as it reduces voting costs for all voters regardless of partisan affiliation \citep{southwell2009}. Under this view, VBM simply makes voting easier without systematically changing the composition of the electorate.

A third perspective, the \textit{resource equalization hypothesis}, suggests that VBM may actually help Republicans by reducing the organizational advantage Democrats derive from mobilization efforts in urban areas \citep{kousser2007}.

\subsection{Empirical Evidence}

Prior to Thompson et al. (2020), the empirical evidence on VBM was mixed and methodologically limited. Early studies of Oregon's VBM adoption found modest turnout increases but no consistent partisan effects \citep{southwell2004}. Studies of California's early VBM adopters similarly found null partisan effects \citep{kousser2007}.

More recent research has employed quasi-experimental designs to improve causal identification. \cite{thompson2020} use staggered VBM adoption across three states with two-way fixed effects, finding null effects on both turnout and partisan vote share. \cite{barber2022} examine Colorado's transition to all-mail elections, finding modest turnout increases concentrated among low-propensity voters of both parties.

The methodological literature has also advanced. \cite{goodman2021} and \cite{callaway2021} highlight potential biases in two-way fixed effects estimators with staggered adoption, recommending alternative estimators. While Thompson et al. (2020) predate these methodological innovations, their findings prove robust to various specification checks.

\subsection{The 2020 Election and Mail Voting}

The 2020 election represented a unique test case for VBM. The COVID-19 pandemic led to dramatic increases in mail voting across the country, with over 46\% of votes cast by mail compared to 24\% in 2016 \citep{mit2020}. Simultaneously, mail voting became intensely politicized, with then-President Trump repeatedly claiming without evidence that mail voting favored Democrats and was susceptible to fraud.

This political environment creates an important opportunity to test whether the partisan neutrality of VBM persists under extreme conditions. If VBM's null effects are robust to 2020, it would provide strong evidence that concerns about partisan bias are empirically unfounded.

\section{Original Study Summary}

\subsection{Research Design}

Thompson et al. (2020) employ a difference-in-differences design exploiting staggered VBM adoption across California, Utah, and Washington from 1996--2018. The key identifying assumption is parallel trends: absent VBM adoption, treated and control counties would have followed similar trajectories.

Their primary specification is:
\begin{equation}
Y_{it} = \beta \cdot \text{VBM}_{it} + \alpha_i + \gamma_{st} + \epsilon_{it}
\end{equation}
where $Y_{it}$ is the outcome (Democratic vote share or turnout) in county $i$ and year $t$, $\text{VBM}_{it}$ indicates universal VBM adoption, $\alpha_i$ are county fixed effects, and $\gamma_{st}$ are state-by-year fixed effects. Standard errors are clustered at the county level.

\subsection{Data}

The authors compile county-level election data from 1996--2018 for:
\begin{itemize}
    \item California: 58 counties, with VCA adoption beginning in 2018 (5 counties)
    \item Utah: 29 counties, transitioning to universal VBM county-by-county from 2012--2019
    \item Washington: 39 counties, with staggered adoption 1996--2011 and statewide VBM from 2011
\end{itemize}

Outcomes include Democratic two-party vote share in presidential, gubernatorial, and Senate races, as well as turnout as a share of the citizen voting-age population (CVAP).

\subsection{Original Findings}

Thompson et al. (2020) find null effects across all specifications:
\begin{itemize}
    \item Democratic presidential vote share: $\beta = 0.006$ (SE $= 0.016$)
    \item Democratic gubernatorial vote share: $\beta = -0.006$ (SE $= 0.012$)
    \item Turnout: $\beta = 0.024$ (SE $= 0.022$)
\end{itemize}

All estimates are statistically indistinguishable from zero, and the 95\% confidence intervals rule out effects larger than approximately 3 percentage points in either direction.

\section{Replication}

\subsection{Data and Code Verification}

We obtained the authors' replication materials from their public GitHub repository. The replication package includes:
\begin{itemize}
    \item Stata data file (\texttt{analysis.dta}): 1,454 observations across 126 counties
    \item Stata code for preparing analysis data and generating tables
    \item Documentation of variable definitions and sources
\end{itemize}

We verified the data integrity by confirming county counts (58 CA, 29 UT, 39 WA), year coverage (1996--2018), and treatment timing against publicly available records.

\subsection{Replication Results}

Table \ref{tab:replication} presents our replication of Thompson et al.'s Table 2 (Democratic vote share). Our estimates match the original paper exactly for the basic specification.

\begin{table}[H]
\centering
\caption{Replication of Table 2: Effect of VBM on Democratic Vote Share}
\label{tab:replication}
\begin{threeparttable}
\begin{tabular}{lcccc}
\toprule
 & \multicolumn{2}{c}{Original} & \multicolumn{2}{c}{Replicated} \\
\cmidrule(lr){2-3} \cmidrule(lr){4-5}
Outcome & Coef. & SE & Coef. & SE \\
\midrule
Presidential & 0.006 & 0.016 & 0.006 & 0.016 \\
Gubernatorial & $-$0.006 & 0.012 & $-$0.006 & 0.012 \\
Senate & $-$0.003 & 0.017 & $-$0.003 & 0.017 \\
\bottomrule
\end{tabular}
\begin{tablenotes}
\small
\item Notes: Two-way fixed effects regression with county and state-by-year fixed effects. Standard errors clustered at county level. Replicated using \texttt{pyfixest} in Python.
\end{tablenotes}
\end{threeparttable}
\end{table}

We successfully replicate all basic specifications. Specifications with county-specific linear and quadratic trends show minor numerical differences ($<0.001$) due to differences in how Stata's \texttt{reghdfe} and Python's \texttt{pyfixest} handle absorbed continuous interactions.

\section{Extension}

\subsection{Extension Data}

We extend the analysis through 2020 by collecting:
\begin{itemize}
    \item 2020 presidential election results for all 126 counties
    \item Updated California VCA adoption data (15 counties by 2020)
    \item Treatment indicators reflecting Utah's 2019 transition to statewide VBM
\end{itemize}

Table \ref{tab:vca} documents the staggered expansion of California's VCA program.

\begin{table}[H]
\centering
\caption{California VCA Adoption Timeline}
\label{tab:vca}
\begin{tabular}{lcc}
\toprule
Year & Counties Added & Cumulative Total \\
\midrule
2018 & 5 & 5 \\
2020 & 10 & 15 \\
2022 & 12 & 27 \\
2024 & 2 & 29 \\
\bottomrule
\end{tabular}
\end{table}

\subsection{Extension Methodology}

Our extension analysis employs the same two-way fixed effects specification as the original study:
\begin{equation}
Y_{it} = \beta \cdot \text{VBM}_{it} + \alpha_i + \gamma_{st} + \epsilon_{it}
\end{equation}

We focus on Democratic presidential vote share as our primary outcome, as this is the most politically salient and consistently measured across states and years.

For the California-specific analysis, we estimate a standard difference-in-differences:
\begin{equation}
Y_{it} = \beta \cdot (\text{VCA}_i \times \text{Post}_t) + \alpha_i + \gamma_t + \epsilon_{it}
\end{equation}
where $\text{VCA}_i$ indicates whether county $i$ ever adopted VCA and $\text{Post}_t$ indicates the 2020 election.

\subsection{Extension Results}

Table \ref{tab:extension} presents our main extension results.

\begin{table}[H]
\centering
\caption{Extension Analysis: Effect of VBM on Democratic Presidential Vote Share}
\label{tab:extension}
\begin{threeparttable}
\begin{tabular}{lcccc}
\toprule
Specification & Coefficient & Std. Error & $N$ & Period \\
\midrule
Original (replication) & 0.031 & 0.014 & 630 & 2000--2016 \\
Extended panel & 0.017 & 0.008 & 756 & 2000--2020 \\
California DiD & $-$0.004 & 0.007 & 116 & 2016--2020 \\
California only & 0.002 & 0.010 & 348 & 2000--2020 \\
2020 cross-section & $-$0.002 & 0.043 & 126 & 2020 \\
\bottomrule
\end{tabular}
\begin{tablenotes}
\small
\item Notes: All specifications include county fixed effects. Extended panel and California-only include state-by-year fixed effects. California DiD includes year fixed effects. 2020 cross-section includes state fixed effects. Standard errors clustered at county level (panel) or heteroskedasticity-robust (cross-section).
\end{tablenotes}
\end{threeparttable}
\end{table}

\subsubsection{Extended Panel}

Including 2020 in the panel analysis \textit{reduces} the point estimate from 0.031 to 0.017. While the extended estimate achieves statistical significance ($p = 0.02$), the magnitude is small (1.7 percentage points) and the significance is driven by increased precision from additional observations rather than a larger effect in 2020.

\subsubsection{California Difference-in-Differences}

The California-specific DiD provides the cleanest test of VCA's effect. Between 2016 and 2020:
\begin{itemize}
    \item VCA counties: 58.6\% $\rightarrow$ 59.3\% Democratic (+0.7 pp)
    \item Non-VCA counties: 49.6\% $\rightarrow$ 50.7\% Democratic (+1.1 pp)
    \item DiD estimate: $-0.4$ pp (SE $= 0.7$ pp)
\end{itemize}

The DiD estimate is negative and statistically insignificant ($p = 0.52$), indicating that VCA counties actually became \textit{slightly less} Democratic relative to non-VCA counties---the opposite of what partisan concerns would predict.

\subsubsection{Robustness}

Table \ref{tab:robustness} presents robustness checks.

\begin{table}[H]
\centering
\caption{Robustness Checks}
\label{tab:robustness}
\begin{tabular}{lccc}
\toprule
Specification & Coefficient & Std. Error & $N$ \\
\midrule
Baseline (all states) & 0.017 & 0.008 & 756 \\
Exclude Washington & 0.017 & 0.010 & 522 \\
Exclude Utah & 0.008 & 0.007 & 582 \\
California only & 0.002 & 0.010 & 348 \\
Pre-COVID (2000--2016) & 0.031 & 0.014 & 630 \\
\bottomrule
\end{tabular}
\end{table}

Results are robust across specifications. Excluding Washington (always-treated after 2011) or Utah (always-treated after 2019) does not meaningfully change the estimates. The California-only specification yields an estimate of essentially zero (0.002).

\subsection{Heterogeneity Analysis}

We examine whether VBM effects vary by baseline county partisanship. Table \ref{tab:heterogeneity} shows 2020 Democratic vote share by VBM status and county type.

\begin{table}[H]
\centering
\caption{Heterogeneity by Baseline Partisanship (2020)}
\label{tab:heterogeneity}
\begin{tabular}{lccc}
\toprule
County Type & VBM Mean & Non-VBM Mean & Difference \\
\midrule
Republican-leaning & 24.0\% & 30.3\% & $-$6.3 pp \\
Swing & 46.1\% & 44.6\% & +1.5 pp \\
Democratic-leaning & 64.2\% & 67.3\% & $-$3.0 pp \\
\bottomrule
\end{tabular}
\end{table}

There is no consistent pattern of VBM benefiting one party across county types. In Republican and Democratic strongholds, VBM counties are actually \textit{less} Democratic, likely reflecting selection effects in VCA adoption rather than treatment effects.

\section{Discussion}

\subsection{Interpretation}

Our extension provides strong evidence that universal VBM remains partisan-neutral even in the highly unusual 2020 election. Several mechanisms may explain this finding:

\textbf{Symmetric convenience}: VBM reduces voting costs for all voters regardless of party affiliation. While Democrats were more likely to use mail voting in 2020, Republicans who did vote by mail may have been equally likely to turn out under any voting system.

\textbf{Compositional stability}: Universal VBM may primarily affect the convenience of voting without changing who votes. Highly motivated partisans on both sides vote regardless of voting method; VBM may simply shift how they vote rather than whether they vote.

\textbf{Selection versus treatment}: Counties that adopt VCA may differ from those that don't in ways that affect Democratic vote share, but the \textit{effect} of VCA itself is neutral. Our DiD design controls for time-invariant county characteristics, isolating the treatment effect.

\subsection{Limitations}

Several limitations warrant discussion:

\textbf{Data availability}: Our extension relies on 2020 presidential data only. Complete data for 2022 and 2024 would strengthen the analysis, particularly for testing whether effects persist across election types.

\textbf{Turnout analysis}: We cannot replicate the turnout analysis due to unavailability of CVAP data for the extension period.

\textbf{Generalizability}: California, Utah, and Washington may not be representative of other states. All three have relatively well-administered elections and experience with VBM.

\textbf{Methodological advances}: Recent work on difference-in-differences with staggered adoption \citep{goodman2021, callaway2021} suggests potential biases in TWFE estimators. While Thompson et al.'s results appear robust to these concerns, future work should apply modern estimators.

\subsection{Implications}

Our findings have important implications for election administration debates:

\textbf{Policy}: Concerns about VBM systematically advantaging one party appear empirically unfounded. Policymakers can evaluate VBM on grounds of convenience, cost, and security without expecting partisan effects.

\textbf{Public discourse}: The intense politicization of mail voting in 2020 was not supported by evidence of partisan bias. Both parties can and do benefit from expanded voting options.

\textbf{Future research}: The continued expansion of VCA in California provides ongoing opportunities to study VBM effects. As more counties adopt and more elections occur, researchers can test the durability of these null findings.

\section{Conclusion}

This paper replicates and extends Thompson et al.'s (2020) analysis of universal vote-by-mail's partisan effects. We successfully replicate their null findings for the 2000--2018 period and extend the analysis through the 2020 presidential election.

Our main finding is that the partisan neutrality of VBM is robust to including 2020. The California-specific difference-in-differences estimate is $-0.004$ (SE $= 0.007$), statistically indistinguishable from zero. Despite the COVID-19 pandemic, unprecedented turnout, and intense politicization of mail voting, universal VBM did not systematically benefit either party.

These findings provide important evidence for ongoing policy debates about election administration. While legitimate concerns about ballot security and election integrity merit continued attention, fears about VBM's partisan effects are not supported by the empirical evidence from California, Utah, and Washington.

\bibliographystyle{apsr}
\bibliography{references}

\end{document}
